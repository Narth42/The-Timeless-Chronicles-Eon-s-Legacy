\documentclass{article}
\usepackage{graphicx} % Required for inserting images
\usepackage[french]{babel} % pour dire que le texte est en fran¸cais
\usepackage{a4} % pour la taille
\usepackage[T1]{fontenc} % pour les font postscript
\usepackage[cyr]{aeguill} % Police vectorielle TrueType, guillemets fran¸cais
\usepackage{epsfig} % pour g´erer les images
\usepackage{amsmath, amsthm} % tr`es bon mode math´ematique
\usepackage{amsfonts,amssymb}% permet la definition des ensembles
\usepackage{float} % pour le placement des figure
\usepackage{url} % pour une gestion efficace des url

\title{Cahier des charges}
\author{The Timeless Chronicles: Eon's Legacy}
\date{}

\begin{document}

\maketitle
\begin{center}
    Noah Matthieu Abi Chahla\\
    Otto Debrie\\
    Corentin del Pozo\\
    Nathan Hirth\\
    Ilyann Gwinner\\
\end{center}
\pagebreak
\tableofcontents
\pagebreak

\section{Introduction}
<A remplir>
\section{Présentation générale du projet}
<A remplir>
\section{Le jeu}
\subsection{Le but du jeu}
Le jeu se déroule en équipes de 1 à 4 joueurs en ligne, offrant une expérience collaborative très importante dans le déroulement du jeu. Chaque niveau présente un double défi : tout d'abord, il s'agit d'explorer méticuleusement la carte à la recherche du passage menant à la salle finale, puis de coopérer avec d'autres joueurs pour affronter les monstres disséminés à divers endroits sur la carte. Ces affrontements sont cruciaux pour renforcer les joueurs, les préparant ainsi aux épreuves futures. Une fois cette étape franchie, les joueurs doivent affronter un boss dont la principale mission est de protéger le portail permettant d'accéder au niveau suivant.
\subsection{Le gameplay}
<A remplir>
\subsection{L'interface}
\textbf{L'interface est divisée en six zones distinctes:}
\begin{description}
\item[Zone 1] Elle est située en bas à gauche et elle affiche les points de vie, ainsi que la jauge d'énergie, qui équivaut à la barre d'endurance pour certains personnages. De plus, c'est dans cette zone que sera affiché votre pseudo choisi en début de partie.
\item[Zone 2] Positionnée en bas à droite elle est réservée à l'affichage d'informations concernant votre personnage, telles que la classe que vous avez choisie, les améliorations que vous lui avez apportées, et, pour certaines classes, les différents "sorts" dont vous disposez.

\item[Zone 3] Située au milieu en haut elle est dédiée à l'affichage du nombre de points d'expérience acquis par l'équipe. 

\item[Zone 4] Elle se trouve dans le coin supérieur droit et sert à afficher la somme d'argent collectée au cours de la partie.

\item[Zone 5] Elle est en haut à droite et abrite le chat qui permet de communiquer avec les autres joueurs.
\end{description}
\subsection{Pnj}
Dans le jeu, la présence de PNJ sera relativement limitée. Il y aura un marchand qui offrira ses services en vendant des objets aux joueurs en échange de leur argent, tout en étant capable d'engager des dialogues avec eux. Un autre PNJ remplira le rôle de guide, fournissant aux joueurs des informations sur l'histoire du jeu et leur donnant des conseils pour les aider dans leur progression.
\subsection{IA}
<A remplir>
\subsection{Réseau}
<A remplir>
\section{Aspects économiques}
<A remplir>
\section{Découpage du projet}
<A remplir>
\section{Conclusion}
<A remplire>

\end{document}

